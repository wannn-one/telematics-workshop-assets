% Bagian Analisis Hasil Percobaan
\section*{Analisis Hasil Percobaan}

Pada praktikum modul 2 dengan judul Dasar Telematika, praktikan diminta untuk melakukan teknik desoldering, soldering, dan soldering uap. praktikan bersama rekan praktikan langsung berbagi tugas karena sudah membawa solder sendiri. Praktikan menyolder SMD dengan solder uap, sedangkan dengan rekan praktikan melakukan soldering 7 segment. Kami sama sama melakukan desoldering saat melakukan kesalahan dalam proses soldering. Pada proses soldering, kami melumuri bagian yang ingin disolder dengan \textit{flux} terlebih dahulu agar PCB tidak gosong dan jalur solder lebih rapih. Setelah itu kami memanaskan solder hingga mencapai 350°C - 400°C. Timah ditempelkan ke ujung solder dan didekatkan ke PCB agar jalur timah terbentuk. Proses ini terus dilanjutkan hingga jalur rangkaian lengkap.

Pada penyolderan SMD, kami menggunakan timah yang berbeda, yaitu timah cair, yang mana nanti akan mengering dan melekat ketika dipanaskan dengan solder uap. Pada proses solder uap, praktikan mengoleskan timah cair pada kaki pin dengan tujuan agar kaki IC melekat pada tempatnya. Kemudian, setelah semua diolesi, baru kami menempelkan masing masing komponen ke tempatnya, praktikan membutuhkan bantuan pinset untuk memasang komponen pada tempatnya. Setelah semua komponen terpasang, kami memanaskan solder uap hingga mencapai 350°C - 400°C. Setelah mencapai suhu tersebut, praktikan mengipasi PCB dengan solder uap agar timah mengering dan membuat komponen melekat pada PCB. Setelah itu, praktikan membiarkan PCB hingga suhu ruangan dan mengecek apakah komponen sudah melekat dengan baik atau tidak. Jika komponen belum melekat, praktikan akan melakukan proses soldering uap kembali.

Pada praktikum kali ini, kendala yang didapatkan adalah pada saat melakukan soldering uap. Praktikan sempat kehilangan 1 kapasitor SMD, yang mana kapasitor tersebut jatuh ke lantai saat dikeluarkan dari plastiknya. Praktikan juga sempat kesulitan dalam memasang komponen SMD pada tempatnya karena ukuran komponen yang kecil.

Hasil dari praktikum kali ini adalah PCB yang sudah terpasang komponen SMD dan 7 segment. 7 \textit{segment display} sudah bisa mengeluarkan 'C' dan 'E', kemudian SMD sudah bisa mengeluarkan LED Chaser dengan 10 LED. Saat diputar melawan jarum jam, makan kecepatan LED Chaser akan melambat, sedangkan jika diputar searah jarum jam, maka kecepatan LED Chaser akan mempercepat.

\newpage