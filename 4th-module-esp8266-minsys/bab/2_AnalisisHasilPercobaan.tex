% Bagian Analisis Hasil Percobaan
\section*{Analisis Hasil Percobaan}
\indent
Pada praktikum ke 4 tentang ESP8266 Minimum System ini, saya berhasil melakukan percobaan yang ada pada modul praktikum. Berikut adalah hasil percobaan yang saya lakukan:

\begin{enumerate}
    \item Stencil dan Aplikasi Pasta Solder
    \item Penempatan Komponen dan Penyolderan
    \item Upload Firmware dan Pengujian Fungsional
    \item Pengujian Program Kelompok
\end{enumerate}

\indent
Pada percobaan pertama, saya mengoleskan pasta solder pada PCB yang telah ditempelkan stencil. Kemudian saya mengangkat stencil dan menempelkan komponen yang telah disiapkan pada PCB. Penempelan harus dilakukan secara hati-hati agar komponen tidak tercecer. Setelah itu, saya melakukan penyolderan pada komponen yang telah ditempelkan. Penyolderan dilakukan dengan menggunakan solder dan heated bed. Setelah semua komponen tersolder, saya melakukan pengujian fungsional dengan cara menghubungkan ESP8266 Minimum System dengan laptop menggunakan kabel USB TTL.

\indent
Setelah itu, saya melakukan percobaan ketiga. Pada percobaan ketiga, saya melakukan upload firmware dan pengujian fungsional. Upload firmware dilakukan dengan menggunakan ekstensi PlatformIO. Setelah firmware berhasil diupload, kami mencoba mengupload program yang telah dibuat oleh kelompok kami. Program yang kami buat adalah program untuk mengontrol LED yang terhubung dengan ESP8266 Minimum System. Program tersebut dapat diakses melalui jaringan WiFi yang dibuat oleh ESP8266 Minimum System. Setelah program berhasil diupload, kami mencoba mengakses program tersebut melalui jaringan WiFi yang dibuat oleh ESP8266 Minimum System. Program tersebut dapat diakses dengan menggunakan browser. Setelah program berhasil diakses, kami mencoba mengontrol LED yang terhubung dengan ESP8266 Minimum System. Program yang kami buat berhasil dijalankan dengan baik.
% \begin{table}[h]
%     \centering
%     \caption{Caption tabelnya}
%     \label{tab:labelini}
%     \begin{tabular}{|c|c|c|}
%     \hline
%     Kolom 1 & Kolom 2 & Kolom 3 \\
%     \hline
%     Data 1 & Data 2 & Data 3 \\
%     Data 4 & Data 5 & Data 6 \\
%     \hline
%     \end{tabular}
% \end{table}